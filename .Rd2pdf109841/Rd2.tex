\nonstopmode{}
\documentclass[letterpaper]{book}
\usepackage[times,inconsolata,hyper]{Rd}
\usepackage{makeidx}
\usepackage[utf8]{inputenc} % @SET ENCODING@
% \usepackage{graphicx} % @USE GRAPHICX@
\makeindex{}
\begin{document}
\chapter*{}
\begin{center}
{\textbf{\huge MeTime R Package}}
\par\bigskip{\large \today}
\end{center}
\inputencoding{utf8}
\HeaderA{add\_clusters\_wgcna}{Function to add the clusters obtained from wgcna}{add.Rul.clusters.Rul.wgcna}
%
\begin{Description}\relax
A function to add cluster assiginment to the col\_data from WGCNA
\end{Description}
%
\begin{Usage}
\begin{verbatim}
add_clusters_wgcna(object, which_data, baseline, ...)
\end{verbatim}
\end{Usage}
%
\begin{Arguments}
\begin{ldescription}
\item[\code{object}] An S4 object of class metime\_analyser

\item[\code{which\_data}] character to define which dataset is to be used

\item[\code{baseline}] character to define the timepoint to be used as baseline to predict clusters

\item[\code{...}] other parameters for cutreeDynamic function such minClusterSize, pamDendroRespect etc
\end{ldescription}
\end{Arguments}
%
\begin{Value}
metime\_analyser object with updated column info about the clustersize
\end{Value}
\inputencoding{utf8}
\HeaderA{add\_col\_stats}{Function to check normality and add data to col data}{add.Rul.col.Rul.stats}
%
\begin{Description}\relax
A method applied on the s4 object of class "metime\_analyser" to check normality of the metabolites
and add it to corresponding columns
\end{Description}
%
\begin{Usage}
\begin{verbatim}
add_col_stats(object, which_data, type, metab_names, all)
\end{verbatim}
\end{Usage}
%
\begin{Arguments}
\begin{ldescription}
\item[\code{object}] An object of class metime\_analyser

\item[\code{which\_data}] dataset on which the method is to be applied

\item[\code{type}] type of test, "shapiro" and "kruskal" are available

\item[\code{metab\_names}] column that has the metabolite names in col\_data.

\item[\code{all}] logical to add all kinds of available stats.
\end{ldescription}
\end{Arguments}
%
\begin{Value}
S4 object with shapiro wilk test related data in the col\_data
\end{Value}
%
\begin{Examples}
\begin{ExampleCode}
object <- add_col_normality(object=data, which_data=c("lipid_data","nmr_data"), type="shapiro", metab_names=c("metabolite","id"))
\end{ExampleCode}
\end{Examples}
\inputencoding{utf8}
\HeaderA{add\_distribution\_vars\_to\_rows}{Function to add measurements taken at screening time for samples to be added to all timepoints in row data}{add.Rul.distribution.Rul.vars.Rul.to.Rul.rows}
%
\begin{Description}\relax
A method applied on the s4 object of class "metime\_analyser" to add all those datapoints that were measured only during screening
to all the respective samples at all timepoints in row\_data lists
\end{Description}
%
\begin{Usage}
\begin{verbatim}
add_distribution_vars_to_rows(
  object,
  screening_vars,
  distribution_vars,
  which_data
)
\end{verbatim}
\end{Usage}
%
\begin{Arguments}
\begin{ldescription}
\item[\code{object}] An object of class metime\_analyser

\item[\code{screening\_vars}] Logical to call add\_screening\_vars() before updating rows

\item[\code{distribution\_vars}] A character naming the vars of interest

\item[\code{which\_data}] dataset to which the information is to be added(only 1 can be used at a time)
\end{ldescription}
\end{Arguments}
%
\begin{Value}
object of class metime\_analyser with phenotype data added to row data
\end{Value}
%
\begin{Examples}
\begin{ExampleCode}
# adding APOEGrp, PTGENDER, and diag group to all data points and prepping the object for viz_distribution_plotter()
object <- add_distribution_vars_to_rows(object=data, screening_vars=c("APOEGrp", "DXGrp_longi", "PTGENDER"), 
		distribution_vars=c("Age", "BMI", "ADNI_MEM", "ADNI_LAN", "ADNI_EF", "APOEGrp", "DXGrp_longi", "PTGENDER"), which_data="lipid_data")
\end{ExampleCode}
\end{Examples}
\inputencoding{utf8}
\HeaderA{add\_function\_info}{Function to add information of function added to the data}{add.Rul.function.Rul.info}
%
\begin{Description}\relax
Function to add information about the method applied to the dataset
\end{Description}
%
\begin{Usage}
\begin{verbatim}
add_function_info(object, function_name, params)
\end{verbatim}
\end{Usage}
%
\begin{Arguments}
\begin{ldescription}
\item[\code{object}] S4 object of class metime\_analyser

\item[\code{function\_name}] name of the function used

\item[\code{params}] other parameters used wrapped as a list
\end{ldescription}
\end{Arguments}
%
\begin{Value}
object of class metime\_analyser with the information of method applied
\end{Value}
\inputencoding{utf8}
\HeaderA{add\_metabs\_as\_covariates}{Function to add metabolites as covariates for network construction}{add.Rul.metabs.Rul.as.Rul.covariates}
%
\begin{Description}\relax
Method applied on metime\_analyser object to add other metabolite data to a certain dataset
\end{Description}
%
\begin{Usage}
\begin{verbatim}
add_metabs_as_covariates(object, which_data, which_metabs)
\end{verbatim}
\end{Usage}
%
\begin{Arguments}
\begin{ldescription}
\item[\code{object}] A S4 object of class metime\_analyser

\item[\code{which\_data}] Dataset to which the metab data is to be added(please note that this a single character)

\item[\code{which\_metabs}] list of names of metabs and name of the list represents the dataset from which
the metabs are to be acquired. eg: which\_metabs=list(nmr\_data=c("metab1", "metab2"), lipid\_data=c(""))
\end{ldescription}
\end{Arguments}
%
\begin{Value}
S4 object with metabs added for GGM to another dataset
\end{Value}
\inputencoding{utf8}
\HeaderA{add\_node\_features}{Function to add features to visnetwork plot from another plotter object}{add.Rul.node.Rul.features}
%
\begin{Description}\relax
Function to add node features to see the nodes in the network that affected differently
\end{Description}
%
\begin{Usage}
\begin{verbatim}
add_node_features(
  network_plotter_object,
  guide_object,
  which_type,
  metab_colname
)
\end{verbatim}
\end{Usage}
%
\begin{Arguments}
\begin{ldescription}
\item[\code{network\_plotter\_object}] plotter object with network information

\item[\code{guide\_object}] guide from which the colors are to be extracted. Both analyser and plotter objects are allowed

\item[\code{which\_type}] type of the guide plotter object to be used. Current options are c("regression","conservation")

\item[\code{metab\_colname}] name of the column in guide plotter object that represents the metabolites
\end{ldescription}
\end{Arguments}
%
\begin{Value}
network plotter object with new node colors/features
\end{Value}
\inputencoding{utf8}
\HeaderA{add\_phenotypes\_as\_covariates}{Function to add covariates to the dataset of interest for GGMs}{add.Rul.phenotypes.Rul.as.Rul.covariates}
%
\begin{Description}\relax
adds Covariates to data matrices in metime\_analyser S4 object
\end{Description}
%
\begin{Usage}
\begin{verbatim}
add_phenotypes_as_covariates(
  object,
  which_data,
  covariates,
  class.ind,
  phenotype
)
\end{verbatim}
\end{Usage}
%
\begin{Arguments}
\begin{ldescription}
\item[\code{object}] object of class metime\_analyser

\item[\code{which\_data}] Dataset to which the covariates is to be added

\item[\code{covariates}] character vector names of covariates.

\item[\code{class.ind}] Logical to convert factor variables into class.ind style or not

\item[\code{phenotype}] Logical. If True will extract from phenotype dataset else uses row data
\end{ldescription}
\end{Arguments}
%
\begin{Value}
S4 object with covariates added to the dataset
\end{Value}
\inputencoding{utf8}
\HeaderA{add\_screening\_vars}{Function to add measurements taken at screening time for samples to be added to all timepoints}{add.Rul.screening.Rul.vars}
%
\begin{Description}\relax
A method applied on the s4 object of class "metime\_analyser" to add all those datapoints that were measured only during screening
to all the respective samples at all timepoints
\end{Description}
%
\begin{Usage}
\begin{verbatim}
add_screening_vars(object, vars)
\end{verbatim}
\end{Usage}
%
\begin{Arguments}
\begin{ldescription}
\item[\code{object}] An object of class metime\_analyser

\item[\code{vars}] A character naming the vars of interest
\end{ldescription}
\end{Arguments}
%
\begin{Value}
phenotype data which can be replaced into the original object or use it separately with a different object
\end{Value}
%
\begin{Examples}
\begin{ExampleCode}
# adding APOEGrp, PTGENDER to all data points
new_with_apoegrp_sex <- add_screening_vars(object=metime_analyser_object, vars=c("APOEGrp","PTGENDER"))
\end{ExampleCode}
\end{Examples}
\inputencoding{utf8}
\HeaderA{calc\_colinearity}{Function to calculate colinearity}{calc.Rul.colinearity}
%
\begin{Description}\relax
Function to calculate colinearity in a dataset
\end{Description}
%
\begin{Usage}
\begin{verbatim}
calc_colinearity(
  object,
  which_data,
  cols_for_meta,
  show_all,
  name,
  stratifications
)
\end{verbatim}
\end{Usage}
%
\begin{Arguments}
\begin{ldescription}
\item[\code{object}] An S4 object of class metime\_analyser

\item[\code{which\_data}] Dataset to check for colinearity

\item[\code{cols\_for\_meta}] list of character vectors of column names needed in metadata. Id and class name is needed

\item[\code{show\_all}] logical. True will only filter out colinear data

\item[\code{name}] character to define the name of the result

\item[\code{stratifications}] List to stratify data into a subset. Usage list(name=value)
\end{ldescription}
\end{Arguments}
%
\begin{Value}
plotter object with data for heatmap information
\end{Value}
\inputencoding{utf8}
\HeaderA{calc\_conservation\_metabolite}{Function to calculate metabolite conservation index}{calc.Rul.conservation.Rul.metabolite}
%
\begin{Description}\relax
Method applied on the object metime\_analyser to calculate the metabotype conservation index
\end{Description}
%
\begin{Usage}
\begin{verbatim}
calc_conservation_metabolite(
  object,
  which_data,
  verbose,
  cols_for_meta,
  stratifications,
  name
)
\end{verbatim}
\end{Usage}
%
\begin{Arguments}
\begin{ldescription}
\item[\code{object}] An object of class metime\_analyser

\item[\code{which\_data}] Name of the dataset to be used

\item[\code{verbose}] Information provided on steps being processed

\item[\code{cols\_for\_meta}] A list of a Character vector to define column names that are to be used for plotting purposes

\item[\code{stratifications}] list to stratify the data used

\item[\code{name}] character vector to define the name of the results generated. length should be equal to which\_data
\end{ldescription}
\end{Arguments}
%
\begin{Value}
conservation index results that are added to the object
\end{Value}
%
\begin{Examples}
\begin{ExampleCode}
#calculating metabolite_conservation_index 
out <- calc_metabolite_conservation(object=metime_analyser_object, which_data="Name of the dataset")
\end{ExampleCode}
\end{Examples}
\inputencoding{utf8}
\HeaderA{calc\_conservation\_metabotype}{Function to calculate metabotype conservation index}{calc.Rul.conservation.Rul.metabotype}
%
\begin{Description}\relax
Method applied on the object metime\_analyser to calculate the metabotype conservation index
\end{Description}
%
\begin{Usage}
\begin{verbatim}
calc_conservation_metabotype(
  object,
  which_data,
  timepoints,
  verbose,
  cols_for_meta,
  stratifications,
  name
)
\end{verbatim}
\end{Usage}
%
\begin{Arguments}
\begin{ldescription}
\item[\code{object}] An object of class metime\_analyser

\item[\code{which\_data}] Name of the dataset to be used

\item[\code{timepoints}] character vector with timepoints of interest

\item[\code{verbose}] Information provided on steps being processed

\item[\code{cols\_for\_meta}] Character vector to define column names that are to be used for plotting purposes

\item[\code{stratifications}] List to stratify data into a subset. Usage list(name=value)

\item[\code{name}] character vector to define the results
\end{ldescription}
\end{Arguments}
%
\begin{Value}
List of conservation index results
\end{Value}
%
\begin{Examples}
\begin{ExampleCode}
#calculating metabotype_conservation_index 
out <- calc_metabotype_conservation(object=metime_analyser_object, which_data="Name of the dataset")
\end{ExampleCode}
\end{Examples}
\inputencoding{utf8}
\HeaderA{calc\_correlation\_pairwise}{Function to calculate correlation}{calc.Rul.correlation.Rul.pairwise}
%
\begin{Description}\relax
calculate pairwise correlations
This function creates a dataframe for plotting from a dataset.
\end{Description}
%
\begin{Usage}
\begin{verbatim}
calc_correlation_pairwise(
  object,
  which_data,
  method,
  cols_for_meta,
  name,
  stratifications
)
\end{verbatim}
\end{Usage}
%
\begin{Arguments}
\begin{ldescription}
\item[\code{object}] S4 Object of class metime\_analyser

\item[\code{which\_data}] specify datasets to calculate on. One or more possible

\item[\code{method}] default setting: method="pearson", Alternative "spearman" also possible

\item[\code{cols\_for\_meta}] list equal to length of which\_data defining the columns for metadata

\item[\code{name}] name of the results should be of length=1

\item[\code{stratifications}] List to stratify data into a subset. Usage list(name=value)
\end{ldescription}
\end{Arguments}
%
\begin{Value}
data.frame with pairwise results
\end{Value}
%
\begin{Examples}
\begin{ExampleCode}
# Example to calculate correlations
dist <- calc_correlation(object=metime_analyser_object, which_data="name of the dataset", 
          method="pearson")
\end{ExampleCode}
\end{Examples}
\inputencoding{utf8}
\HeaderA{calc\_dimensionality\_reduction}{Function to calculate dimensionality reduction methods such as tsne, umap and pca.}{calc.Rul.dimensionality.Rul.reduction}
%
\begin{Description}\relax
A method to apply on s4 object of class metime\_analyse in order to obtain information after dimensionality reduction on a dataset/s
\end{Description}
%
\begin{Usage}
\begin{verbatim}
calc_dimensionality_reduction(
  object,
  which_data,
  type,
  cols_for_metabs,
  cols_for_samples,
  stratifications,
  ...
)
\end{verbatim}
\end{Usage}
%
\begin{Arguments}
\begin{ldescription}
\item[\code{object}] An object of class metime\_analyser

\item[\code{which\_data}] a character vector - Names of the dataset from which the samples will be extracted

\item[\code{type}] type of the dimensionality reduction method to be applied. Accepted inputs are "UMAP", "tSNE", "PCA"

\item[\code{cols\_for\_metabs}] a list of character vectors for getting metadata for columns for plotting purposes

\item[\code{cols\_for\_samples}] a character vector to define the columns to extract metadata for plotting purposes

\item[\code{stratifications}] List to stratify data into a subset. Usage list(name=value)

\item[\code{...}] additional arguments that can be passed on to prcomp(), M3C::tsne() and umap::umap()
\end{ldescription}
\end{Arguments}
%
\begin{Value}
a list with two plotter objects containing the dimensionality reduction information that can be parsed into plotting function
1) samples - data of the individuals(".\$samples")
2) metabs - data of the metabolites(".\$metabs")
\end{Value}
%
\begin{Examples}
\begin{ExampleCode}
#calculate PCA
pca <- calc_dimensionality_reduction(object=metime_analyser_object, which_data="name/s of the dataset/s", type="PCA")
#calculate UMAP
pca <- calc_dimensionality_reduction(object=metime_analyser_object, which_data="name/s of the dataset/s", type="UMAP")
#calculate tSNE
pca <- calc_dimensionality_reduction(object=metime_analyser_object, which_data="name/s of the dataset/s", type="tSNE")
\end{ExampleCode}
\end{Examples}
\inputencoding{utf8}
\HeaderA{calc\_distance\_pairwise}{Function to calculate dissimilarity using distance measures}{calc.Rul.distance.Rul.pairwise}
%
\begin{Description}\relax
calculate pairwise distances
This function creates a dataframe for plotting from a dataset.
\end{Description}
%
\begin{Usage}
\begin{verbatim}
calc_distance_pairwise(
  object,
  which_data,
  method,
  name,
  cols_for_meta,
  stratifications
)
\end{verbatim}
\end{Usage}
%
\begin{Arguments}
\begin{ldescription}
\item[\code{object}] S4 Object of class metime\_analyser

\item[\code{which\_data}] specify datasets to calculate on. One or more possible

\item[\code{method}] default setting: method="euclidean", Alternative "maximum","minimum",
"manhattan","canberra","minkowski" are also possible

\item[\code{name}] name of the results should be of length=1

\item[\code{cols\_for\_meta}] list equal to length of which\_data defining the columns for metadata

\item[\code{stratifications}] List to stratify data into a subset. Usage list(name=value)
\end{ldescription}
\end{Arguments}
%
\begin{Value}
data.frame with pairwise results
\end{Value}
%
\begin{Examples}
\begin{ExampleCode}
# Example to calculate pairwise distances
dist <- calc_pairwise_distance(object=metime_analyser_object, which_data="name of the dataset", 
          method="euclidean")
\end{ExampleCode}
\end{Examples}
\inputencoding{utf8}
\HeaderA{calc\_featureselection\_boruta}{Function to calculate dependent variables}{calc.Rul.featureselection.Rul.boruta}
%
\begin{Description}\relax
An S4 method to be applied on the metime\_analyser object so as to calculate dependent variables
\end{Description}
%
\begin{Usage}
\begin{verbatim}
calc_featureselection_boruta(
  object,
  which_x,
  which_y,
  verbose,
  output_loc,
  file_name
)
\end{verbatim}
\end{Usage}
%
\begin{Arguments}
\begin{ldescription}
\item[\code{object}] An object of class metime\_analyser

\item[\code{which\_x}] Name of the dataset to be used for training

\item[\code{which\_y}] Name of the dataset to be used for testing

\item[\code{verbose}] Information provided on steps being processed

\item[\code{output\_loc}] path to the parent directory where in the out file wíll be stored

\item[\code{file\_name}] name of the out file
\end{ldescription}
\end{Arguments}
%
\begin{Value}
List of conservation index results
\end{Value}
\inputencoding{utf8}
\HeaderA{calc\_ggm\_genenet\_longitudnal}{An automated fucntion to calculate GGM from genenet longitudnal version}{calc.Rul.ggm.Rul.genenet.Rul.longitudnal}
%
\begin{Description}\relax
automated funtion that can be applied on metime\_analyser object to obtain geneNet network along with threshold used
\end{Description}
%
\begin{Usage}
\begin{verbatim}
calc_ggm_genenet_longitudnal(
  object,
  which_data,
  threshold,
  all,
  cols_for_meta,
  covariates,
  stratifications,
  name,
  ...
)
\end{verbatim}
\end{Usage}
%
\begin{Arguments}
\begin{ldescription}
\item[\code{object}] S4 object of cĺass metime\_analyser

\item[\code{which\_data}] a character or a character vector naming the datasets of interest

\item[\code{threshold}] type of threshold to be used for extracting significant edge

\item[\code{all}] Logical to get all edges without any cutoff.

\item[\code{cols\_for\_meta}] a list of character vectors for getting metadata for columns for plotting purposes

\item[\code{covariates}] covariates to be used for this analysis

\item[\code{stratifications}] List to stratify data into a subset. Usage list(name=value)

\item[\code{name}] character vector for naming the results

\item[\code{...}] addtional arguments for genenet network
\end{ldescription}
\end{Arguments}
%
\begin{Value}
Network data as a plotter object
\end{Value}
\inputencoding{utf8}
\HeaderA{calc\_ggm\_genenet}{An automated fucntion to calculate GGM from genenet crosssectional version}{calc.Rul.ggm.Rul.genenet}
%
\begin{Description}\relax
automated funtion that can be applied on metime\_analyser object to obtain geneNet network along with threshold used
\end{Description}
%
\begin{Usage}
\begin{verbatim}
calc_ggm_genenet(
  object,
  which_data,
  threshold,
  all,
  cols_for_meta,
  covariates,
  stratifications,
  name,
  ...
)
\end{verbatim}
\end{Usage}
%
\begin{Arguments}
\begin{ldescription}
\item[\code{object}] S4 object of cĺass metime\_analyser

\item[\code{which\_data}] a character or a character vector naming the datasets of interest

\item[\code{threshold}] type of threshold to be used for extracting significant edges.
allowed inputs are "li", "FDR", "bonferroni"

\item[\code{all}] Logical to extract all edges without any pval correction

\item[\code{cols\_for\_meta}] list of character vector for extracting metadata of metabolites for plotting

\item[\code{covariates}] covariates to be used for this analysis

\item[\code{stratifications}] List to stratify data into a subset. Usage list(name=value)

\item[\code{name}] Name of the result

\item[\code{...}] additional arguments for GeneNet
\end{ldescription}
\end{Arguments}
%
\begin{Value}
Network data as a plotter object
\end{Value}
\inputencoding{utf8}
\HeaderA{calc\_ggm\_multibipartite\_lasso}{An automated fucntion to calculate GGM from multibipartite lasso approach}{calc.Rul.ggm.Rul.multibipartite.Rul.lasso}
%
\begin{Description}\relax
automated funtion that can be applied on s4 object of class metime\_analyser to calculate a network using
multibipartite lasso
\end{Description}
%
\begin{Usage}
\begin{verbatim}
calc_ggm_multibipartite_lasso(
  object,
  which_data,
  alpha,
  nfolds,
  timepoints,
  cols_for_meta
)
\end{verbatim}
\end{Usage}
%
\begin{Arguments}
\begin{ldescription}
\item[\code{object}] S4 object of cĺass metime\_analyser

\item[\code{which\_data}] a character or a character vector naming the datasets of interest

\item[\code{alpha}] tuning parameter for lasso + ridge regression in glmnet

\item[\code{nfolds}] nfolds for cv.glmnet

\item[\code{timepoints}] timepoints of interest that are to be used to build networks(as per timepoints in rows)

\item[\code{cols\_for\_meta}] a list of character vectors of column names to be used for visualization of the networks.
\end{ldescription}
\end{Arguments}
%
\begin{Value}
list of plotter objects that can be used for plotting.
\end{Value}
\inputencoding{utf8}
\HeaderA{calc\_lm\_matrixeqtl}{Function to perform matrixEQTL style regression longitudinally}{calc.Rul.lm.Rul.matrixeqtl}
%
\begin{Description}\relax
Function to perform matrixEQTL style regression longitudinally
\end{Description}
%
\begin{Usage}
\begin{verbatim}
calc_lm_matrixeqtl(
  object,
  which_data,
  covariates,
  feature_selection,
  regression_phenotype,
  name,
  stratifications,
  cols_for_meta
)
\end{verbatim}
\end{Usage}
%
\begin{Arguments}
\begin{ldescription}
\item[\code{object}] An S4 object of class metime\_analyser

\item[\code{which\_data}] character to define dataset to be used

\item[\code{covariates}] covariates to be used

\item[\code{feature\_selection}] results from feature\_selection

\item[\code{regression\_phenotype}] character to define which phenotype

\item[\code{name}] character vector for naming the results

\item[\code{stratifications}] List to stratify data into a subset. Usage list(name=value)

\item[\code{cols\_for\_meta}] colnames to be added for meta information. list of character vectors of length which\_data
\end{ldescription}
\end{Arguments}
%
\begin{Value}
plotter object with a forest plot
\end{Value}
\inputencoding{utf8}
\HeaderA{calc\_parafac}{Function to perform PARAFAC analysis}{calc.Rul.parafac}
%
\begin{Description}\relax
Method to be applied on S4 object of class metime\_analyser to perform PARAFAC analysis
\end{Description}
%
\begin{Usage}
\begin{verbatim}
calc_parafac(object, which_data, timepoints, nfac = 3, ...)
\end{verbatim}
\end{Usage}
%
\begin{Arguments}
\begin{ldescription}
\item[\code{object}] S4 object of class metime\_analyser

\item[\code{which\_data}] character vector for dataset to be used

\item[\code{timepoints}] character vector to define timepoints of interest

\item[\code{nfac}] parameter nfac for parafac(). Numeric value to define the number of factors. Default is set to 3

\item[\code{...}] Additional arguments to be used for the function parafac()
\end{ldescription}
\end{Arguments}
%
\begin{Value}
An object of class PARAFAC. See multiway library for more information
\end{Value}
\inputencoding{utf8}
\HeaderA{calc\_temporal\_ggm}{An automated function to caluclate temporal network with lagged model}{calc.Rul.temporal.Rul.ggm}
%
\begin{Description}\relax
calculates temporal networks for each dataset with a lagged model as used in graphical VAR
\end{Description}
%
\begin{Usage}
\begin{verbatim}
calc_temporal_ggm(
  object,
  which_data,
  lag,
  timepoints,
  alpha,
  nfolds,
  cols_for_meta,
  cores
)
\end{verbatim}
\end{Usage}
%
\begin{Arguments}
\begin{ldescription}
\item[\code{object}] S4 object of class metab\_analyser

\item[\code{which\_data}] dataset or datasets to be used

\item[\code{lag}] which lagged model to use. 1 means one-lagged model, similary 2,3,..etc

\item[\code{timepoints}] timepoints of interest that are to be used to build networks(in the order of measurement)

\item[\code{alpha}] parameter for regression coefficient

\item[\code{nfolds}] nfolds parameter for glmnet style of regression

\item[\code{cols\_for\_meta}] a list of character vectors of column names to be used for visualization of the networks.

\item[\code{cores}] Number of cores to be used for the process
\end{ldescription}
\end{Arguments}
%
\begin{Value}
temporal network data with edgelist and regression values
\end{Value}
\inputencoding{utf8}
\HeaderA{calc\_trajectories\_by\_mean}{Function to get mean trajectories of metabolites and phenotypic traits}{calc.Rul.trajectories.Rul.by.Rul.mean}
%
\begin{Description}\relax
function to extract mean trajectories
\end{Description}
%
\begin{Usage}
\begin{verbatim}
calc_trajectories_by_mean(object, which_data, columns)
\end{verbatim}
\end{Usage}
%
\begin{Arguments}
\begin{ldescription}
\item[\code{object}] An S4 object of class metime\_analyser

\item[\code{which\_data}] Dataset of interest

\item[\code{columns}] Other data that you want to see along with metabolites(column names from rowdata)
\end{ldescription}
\end{Arguments}
%
\begin{Value}
plot\_data table that can be used to make the plotter object without metadata
\end{Value}
\inputencoding{utf8}
\HeaderA{calc\_ttest\_metabolites}{Function to calculate students t-test between metabolites at different timepoints}{calc.Rul.ttest.Rul.metabolites}
%
\begin{Description}\relax
Method for S4 object of class metime\_analyser for performing t-test
\end{Description}
%
\begin{Usage}
\begin{verbatim}
calc_ttest_metabolites(object, which_data, timepoints, split_var, type, paired)
\end{verbatim}
\end{Usage}
%
\begin{Arguments}
\begin{ldescription}
\item[\code{object}] S4 object of class metime\_analyser

\item[\code{which\_data}] dataset or datasets to be used for the analysis

\item[\code{timepoints}] timepoints of interest to perform the test on

\item[\code{split\_var}] split variable for testing such as diagnostic group etc

\item[\code{type}] type of ttest to be used either "two.sided", "less", or "greater"

\item[\code{paired}] Logical to perform paired t.test or not
\end{ldescription}
\end{Arguments}
%
\begin{Value}
plotter object with t-test results
\end{Value}
\inputencoding{utf8}
\HeaderA{calc\_ttest\_samples}{Function to calculate students t-test between samples at different timepoints}{calc.Rul.ttest.Rul.samples}
%
\begin{Description}\relax
Method for S4 object of class metime\_analyser for performing t-test
\end{Description}
%
\begin{Usage}
\begin{verbatim}
calc_ttest_samples(object, which_data, timepoints, type, paired = TRUE)
\end{verbatim}
\end{Usage}
%
\begin{Arguments}
\begin{ldescription}
\item[\code{object}] S4 object of class metime\_analyser

\item[\code{which\_data}] dataset or datasets to be used for the analysis

\item[\code{timepoints}] timepoints of interest to perform the test on

\item[\code{type}] type of ttest to be used either "two.sided", "less", or "greater"

\item[\code{paired}] Logical to perform paired t.test or not
\end{ldescription}
\end{Arguments}
%
\begin{Value}
plotter object with t-test results
\end{Value}
\inputencoding{utf8}
\HeaderA{check\_col\_normality}{Function to check for col\_normality data whether it is added or not.}{check.Rul.col.Rul.normality}
%
\begin{Description}\relax
function to check whether col\_normality data is added to
the object or not
\end{Description}
%
\begin{Usage}
\begin{verbatim}
check_col_normality(object, which_data)
\end{verbatim}
\end{Usage}
%
\begin{Arguments}
\begin{ldescription}
\item[\code{object}] S4 object of class of metime\_analyser

\item[\code{which\_data}] dataset/s to check
\end{ldescription}
\end{Arguments}
%
\begin{Value}
NULL if it passes all the sanity checks
\end{Value}
\inputencoding{utf8}
\HeaderA{check\_ids\_and\_classes}{Function to check the ids in the data and data format}{check.Rul.ids.Rul.and.Rul.classes}
%
\begin{Description}\relax
sanity check to check for ids and order of the data
\end{Description}
%
\begin{Usage}
\begin{verbatim}
check_ids_and_classes(object)
\end{verbatim}
\end{Usage}
%
\begin{Arguments}
\begin{ldescription}
\item[\code{object}] S4 object of class of metime\_analyser
\end{ldescription}
\end{Arguments}
%
\begin{Value}
NULL if it passes all the sanity checks
\end{Value}
\inputencoding{utf8}
\HeaderA{check\_results}{Function to check the format of results if they exist}{check.Rul.results}
%
\begin{Description}\relax
sanity check to check for results of the analysis
\end{Description}
%
\begin{Usage}
\begin{verbatim}
check_results(object)
\end{verbatim}
\end{Usage}
%
\begin{Arguments}
\begin{ldescription}
\item[\code{object}] S4 object of class of metime\_analyser
\end{ldescription}
\end{Arguments}
%
\begin{Value}
NULL if it passes all the sanity checks
\end{Value}
\inputencoding{utf8}
\HeaderA{check\_rownames\_and\_colnames}{Function to check the format of rownames and colnames and if they are same or not}{check.Rul.rownames.Rul.and.Rul.colnames}
%
\begin{Description}\relax
sanity check to check for rownames of the data
\end{Description}
%
\begin{Usage}
\begin{verbatim}
check_rownames_and_colnames(object)
\end{verbatim}
\end{Usage}
%
\begin{Arguments}
\begin{ldescription}
\item[\code{object}] S4 object of class of metime\_analyser
\end{ldescription}
\end{Arguments}
%
\begin{Value}
NULL if it passes all the sanity checks
\end{Value}
\inputencoding{utf8}
\HeaderA{check\_scaling\_and\_transformation}{Function to check if the data is already scaled or log transformed}{check.Rul.scaling.Rul.and.Rul.transformation}
%
\begin{Description}\relax
Function to be applied on metime\_analyser to check for log transformation and scaling
\end{Description}
%
\begin{Usage}
\begin{verbatim}
check_scaling_and_transformation(object, which_data)
\end{verbatim}
\end{Usage}
%
\begin{Arguments}
\begin{ldescription}
\item[\code{object}] An S4 object of metime\_anlyser class

\item[\code{which\_data}] the dataset/s to be checked
\end{ldescription}
\end{Arguments}
%
\begin{Value}
NULL but checks if the data is scaled or not
\end{Value}
\inputencoding{utf8}
\HeaderA{get\_append\_analyser\_object}{This function appends an object of class metime\_analyser with a new dataset.}{get.Rul.append.Rul.analyser.Rul.object}
%
\begin{Description}\relax
function to apply on metime\_analyse object to append a new dataset into the existing object
\end{Description}
%
\begin{Usage}
\begin{verbatim}
get_append_analyser_object(object, data, col_data, row_data, name)
\end{verbatim}
\end{Usage}
%
\begin{Arguments}
\begin{ldescription}
\item[\code{object}] S4 object of class metime\_analyser

\item[\code{data}] data.frame containing data

\item[\code{col\_data}] data.frame containing col\_data: id column of col data has to match colnames of data

\item[\code{row\_data}] data.frame containing row\_data: id column of row data has to match rownames of data

\item[\code{name}] Name of the new dataset
\end{ldescription}
\end{Arguments}
%
\begin{Value}
An object of class metime\_analyser
\end{Value}
%
\begin{Examples}
\begin{ExampleCode}
# append data frames into the metime_analyser object
appended_object <- get_append_metab_object(object=metime_analyser_object, data=data, row_data=data, col_data=col_data, name="name of the new dataset")
\end{ExampleCode}
\end{Examples}
\inputencoding{utf8}
\HeaderA{get\_betas\_for\_multibipartite\_lasso}{Function to perform multibipartite style regression on a list of matrices}{get.Rul.betas.Rul.for.Rul.multibipartite.Rul.lasso}
%
\begin{Description}\relax
Performs multibipartite lasso in cv.glmnet style on a list of matrices that have
metabolite information from different platforms
\end{Description}
%
\begin{Usage}
\begin{verbatim}
get_betas_for_multibipartite_lasso(list_of_mats, alpha, nfolds)
\end{verbatim}
\end{Usage}
%
\begin{Arguments}
\begin{ldescription}
\item[\code{list\_of\_mats}] a list with matrices and samples ordered similarly

\item[\code{alpha}] alpha for cv.glmnet regression. Defines style of penalty.

\item[\code{nfolds}] nfolds for cv.glmnet
\end{ldescription}
\end{Arguments}
%
\begin{Value}
returns a list with information of the combinations in context
\end{Value}
\inputencoding{utf8}
\HeaderA{get\_class\_info\_from\_edges}{Function to get information on how many class edges are present}{get.Rul.class.Rul.info.Rul.from.Rul.edges}
%
\begin{Description}\relax
Function to check how the different edges in a GGM are associated to their
respective classes(it could be super-pathway or sub-pathway)
\end{Description}
%
\begin{Usage}
\begin{verbatim}
get_class_info_from_edges(calc_networks, metadata, phenotypes)
\end{verbatim}
\end{Usage}
%
\begin{Arguments}
\begin{ldescription}
\item[\code{calc\_networks}] list of calculated networks

\item[\code{metadata}] metadata of the edges present

\item[\code{phenotypes}] character vector to define phenotypes that were used for correcting the data
\end{ldescription}
\end{Arguments}
%
\begin{Value}
table with information on different type of edges present
\end{Value}
\inputencoding{utf8}
\HeaderA{get\_coldata}{Function to extract col data of a dataset}{get.Rul.coldata}
%
\begin{Description}\relax
Function to get coldata
\end{Description}
%
\begin{Usage}
\begin{verbatim}
get_coldata(object, which_data)
\end{verbatim}
\end{Usage}
%
\begin{Arguments}
\begin{ldescription}
\item[\code{object}] An object of class S4

\item[\code{which\_data}] Dataset of interest
\end{ldescription}
\end{Arguments}
%
\begin{Value}
col data of the dataset of interest
\end{Value}
\inputencoding{utf8}
\HeaderA{get\_environment}{Function to get the R environment}{get.Rul.environment}
%
\begin{Description}\relax
function to print the R environment
\end{Description}
%
\begin{Usage}
\begin{verbatim}
get_environment()
\end{verbatim}
\end{Usage}
%
\begin{Value}
null
\end{Value}
\inputencoding{utf8}
\HeaderA{get\_files\_and\_names}{Function to pack all the data into a single object of class "metime\_analyser"}{get.Rul.files.Rul.and.Rul.names}
%
\begin{Description}\relax
This function loads all the files from the parent directory. It assumes a
certain naming pattern as follows: "datatype\_None|col|row\_data.rds"
Any other naming pattern is not allowed. The function first writes
all files into a list and each type of data is packed into its respective
class i.e. col\_data, row\_data or data
\end{Description}
%
\begin{Usage}
\begin{verbatim}
get_files_and_names(path, annotations_index)
\end{verbatim}
\end{Usage}
%
\begin{Arguments}
\begin{ldescription}
\item[\code{path}] Path to the parent directory

\item[\code{annotations\_index}] a list to be filled as
list(phenotype="Name or index of the files",
medication="Name or index of the files")
\end{ldescription}
\end{Arguments}
%
\begin{Value}
An object of class metime\_analyser
\end{Value}
%
\begin{Examples}
\begin{ExampleCode}
get_files_and_names(path="/path/to/parent/directory", 
	annotations_index=list(phenotype="Name of phenotype file", 
medication="name of phenotype file"))

\end{ExampleCode}
\end{Examples}
\inputencoding{utf8}
\HeaderA{get\_ggm\_genenet}{Function to calculate a dynamic GeneNet GGM from a longitudnal data matrix}{get.Rul.ggm.Rul.genenet}
%
\begin{Description}\relax
calculates GGM on longitudnal data matrix and returns a dataframe with edges,
partial correlation and associated p-values
\end{Description}
%
\begin{Usage}
\begin{verbatim}
get_ggm_genenet(data, threshold = c("bonferroni", "FDR", "li"), all, ...)
\end{verbatim}
\end{Usage}
%
\begin{Arguments}
\begin{ldescription}
\item[\code{data}] data matrix in a longitudnal format

\item[\code{threshold}] type of multiple hypothesis correction. Available are Bonferoni("bonferroni"),
Benjamini-Hochberg("FDR") and independent tests method("li", also see Li et al ....)

\item[\code{all}] Logical to get all edges without any cutoff.

\item[\code{...}] additional arguments for ggm.estimate.pcor()
\end{ldescription}
\end{Arguments}
%
\begin{Value}
a dataframe with edges, partial correlation and associated p-values
\end{Value}
\inputencoding{utf8}
\HeaderA{get\_li\_thresh}{Function to calculate multiple tests using method described by li}{get.Rul.li.Rul.thresh}
%
\begin{Description}\relax
li test to check for colinearlity and use it for feature selection
\end{Description}
%
\begin{Usage}
\begin{verbatim}
get_li_thresh(object, which_data, verbose)
\end{verbatim}
\end{Usage}
%
\begin{Arguments}
\begin{ldescription}
\item[\code{object}] an S4 object of class metime\_analyser

\item[\code{which\_data}] dataset to be used for testing

\item[\code{verbose}] Logical to print out the number of independent tests
\end{ldescription}
\end{Arguments}
%
\begin{Value}
li threshold value
\end{Value}
\inputencoding{utf8}
\HeaderA{get\_make\_analyser\_object}{Function to pack all the data into a single object of class "metime\_analyser"}{get.Rul.make.Rul.analyser.Rul.object}
%
\begin{Description}\relax
This function creates an object of class metime\_analyser from a dataset.
\end{Description}
%
\begin{Usage}
\begin{verbatim}
get_make_analyser_object(
  data,
  col_data,
  row_data,
  annotations_index = list(),
  name = NULL,
  results = list()
)
\end{verbatim}
\end{Usage}
%
\begin{Arguments}
\begin{ldescription}
\item[\code{data}] data.frame containing data

\item[\code{col\_data}] data.frame containing col\_data: id column of col data has to match colnames of data

\item[\code{row\_data}] data.frame containing row\_data: id column of row data has to match rownames of data

\item[\code{annotations\_index}] a list to be filled as follows = list(phenotype="Name or index of the file/list", medication="Name or index of the files/list")

\item[\code{name}] character. Name you want to assign to the new dataset that is being added on

\item[\code{results}] list set to empty but can add any existing results
\end{ldescription}
\end{Arguments}
%
\begin{Value}
An object of class metime\_analyser
\end{Value}
\inputencoding{utf8}
\HeaderA{get\_make\_results}{Function to make results list for metime\_analyser object}{get.Rul.make.Rul.results}
%
\begin{Description}\relax
function to generate results for metime\_analyser object
\end{Description}
%
\begin{Usage}
\begin{verbatim}
get_make_results(object, data, metadata, calc_type, calc_info, name)
\end{verbatim}
\end{Usage}
%
\begin{Arguments}
\begin{ldescription}
\item[\code{object}] An S4 object of class metime\_analyser

\item[\code{data}] list of dataframes of plotable data obtained from any calc function

\item[\code{metadata}] list of dataframes with the metadata for the plot table mentioned above. To obtain these see
get\_metadata\_for\_rows() and get\_metadata\_for\_columns()

\item[\code{calc\_type}] A character vector to specify type of calculation - will be used for comp\_ functions
For networks the accepted notations are "genenet\_ggm", "multibipartite\_ggm", and "temporal\_network"
sjould be the same length as the list of data provided

\item[\code{calc\_info}] A string to define the information about calculation, should be the same length as the list
data provided

\item[\code{name}] Name of the result
\end{ldescription}
\end{Arguments}
%
\begin{Value}
object with results of the calculation updated
\end{Value}
\inputencoding{utf8}
\HeaderA{get\_metadata\_for\_columns}{Get metadata for columns(in most cases for metabolites)}{get.Rul.metadata.Rul.for.Rul.columns}
%
\begin{Description}\relax
function to generate a metadata list for building the MeTime plotter object
\end{Description}
%
\begin{Usage}
\begin{verbatim}
get_metadata_for_columns(object, which_data, columns, names, index_of_names)
\end{verbatim}
\end{Usage}
%
\begin{Arguments}
\begin{ldescription}
\item[\code{object}] S4 object of class MeTime Analyser

\item[\code{which\_data}] Names of dataset/s to be used

\item[\code{columns}] A list of character vectors for the columns of interest. Length of the list should be
same as length of which\_data

\item[\code{names}] A Character vector with the new names for the columns mentioned above id should always be first in order

\item[\code{index\_of\_names}] character vector to define the name of the column in which names of the variables are stored
\end{ldescription}
\end{Arguments}
%
\begin{Value}
data.frame with metadata information
\end{Value}
\inputencoding{utf8}
\HeaderA{get\_metadata\_for\_rows}{Get metadata for rows(in most cases for samples)}{get.Rul.metadata.Rul.for.Rul.rows}
%
\begin{Description}\relax
function to generate a metadata list for building the MeTime plotter object
\end{Description}
%
\begin{Usage}
\begin{verbatim}
get_metadata_for_rows(object, which_data, columns)
\end{verbatim}
\end{Usage}
%
\begin{Arguments}
\begin{ldescription}
\item[\code{object}] S4 object of class MeTime Analyser

\item[\code{which\_data}] Names of dataset/s to be used

\item[\code{columns}] A list of character vectors for the columns of interest. Length of the list should be
same as length of which\_data
\end{ldescription}
\end{Arguments}
%
\begin{Value}
data.frame with metadata information for rows
\end{Value}
\inputencoding{utf8}
\HeaderA{get\_palette}{Get a palette of "n" distinct colorblind friendly colors}{get.Rul.palette}
%
\begin{Description}\relax
Function to get a palette of distinct colorblind friendly colors, the distinctiveness is determined by the difference in their hue values.
\end{Description}
%
\begin{Usage}
\begin{verbatim}
get_palette(n)
\end{verbatim}
\end{Usage}
%
\begin{Arguments}
\begin{ldescription}
\item[\code{n}] number of colors wanted in the palette
\end{ldescription}
\end{Arguments}
%
\begin{Value}
a color palette vector with colors in the form of hex codes
\end{Value}
%
\begin{Examples}
\begin{ExampleCode}
# colors=get_palette(n=10)
\end{ExampleCode}
\end{Examples}
\inputencoding{utf8}
\HeaderA{get\_parameters\_of\_results}{Get parameters of the results}{get.Rul.parameters.Rul.of.Rul.results}
%
\begin{Description}\relax
Function to get parameters and functions applied to obtain results
\end{Description}
%
\begin{Usage}
\begin{verbatim}
get_parameters_of_results(object, results_index)
\end{verbatim}
\end{Usage}
%
\begin{Arguments}
\begin{ldescription}
\item[\code{object}] An S4 object of class metime\_analyser

\item[\code{results\_index}] name or index to get to the results of interest
\end{ldescription}
\end{Arguments}
%
\begin{Value}
a dataframes with functions(rownames) and parameters(colnames)
\end{Value}
\inputencoding{utf8}
\HeaderA{get\_rowdata}{Function to extract row data of a dataset}{get.Rul.rowdata}
%
\begin{Description}\relax
Function to get rowdata
\end{Description}
%
\begin{Usage}
\begin{verbatim}
get_rowdata(object, which_data)
\end{verbatim}
\end{Usage}
%
\begin{Arguments}
\begin{ldescription}
\item[\code{object}] An object of class S4

\item[\code{which\_data}] Dataset of interest
\end{ldescription}
\end{Arguments}
%
\begin{Value}
row data of the dataset of interest
\end{Value}
\inputencoding{utf8}
\HeaderA{get\_samples\_and\_timepoints}{Function to know the number of timepoints and the total number of samples available at that point}{get.Rul.samples.Rul.and.Rul.timepoints}
%
\begin{Description}\relax
A method applied onto s4 object of class "metime\_analyser" so as to obtain the number of unique samples available
at each timepoint.
\end{Description}
%
\begin{Usage}
\begin{verbatim}
get_samples_and_timepoints(object, which_data)
\end{verbatim}
\end{Usage}
%
\begin{Arguments}
\begin{ldescription}
\item[\code{object}] An object of class metime\_analyser

\item[\code{which\_data}] Name of the dataset in context
\end{ldescription}
\end{Arguments}
%
\begin{Value}
A data table with timepoints and number of samples at each timepoint
\end{Value}
%
\begin{Examples}
\begin{ExampleCode}
# newdata <- get_samples_and_timepoints(object=metime_analyser_object, which_data="Name of dataset of interest")
\end{ExampleCode}
\end{Examples}
\inputencoding{utf8}
\HeaderA{get\_text\_for\_plot}{Function to Obtain textual information for visualization in interactive plots}{get.Rul.text.Rul.for.Rul.plot}
%
\begin{Description}\relax
a standard function to be applied on data matrices or dataframes with the colnames of interest such that the information from
columns is visualized in the interactive plot
\end{Description}
%
\begin{Usage}
\begin{verbatim}
get_text_for_plot(data, colnames)
\end{verbatim}
\end{Usage}
%
\begin{Arguments}
\begin{ldescription}
\item[\code{data}] a dataframe with plotting data along with other variables for visualization

\item[\code{colnames}] a character vector with the names of the variables that you want to see on the plot
\end{ldescription}
\end{Arguments}
%
\begin{Value}
a vector with strings that can be parsed into plot\_ly text.
\end{Value}
%
\begin{Examples}
\begin{ExampleCode}
# text = get_text(data=data.frame, colnames=c("names","of","columns", "of", "interest"))
\end{ExampleCode}
\end{Examples}
\inputencoding{utf8}
\HeaderA{metime\_analyser-class}{Constructor to generate an object of class metime\_analyser. contains slots - list\_of\_data: For the list of all data matrices. - list\_of\_col\_data: list of all the col data files in the same order. - list\_of\_row\_data: list of all the row data files in the same order. - annotations: list with phenotype and medication. Each of which is character that represents the name of the aforementioned dataset types.}{metime.Rul.analyser.Rdash.class}
%
\begin{Description}\relax
Constructor to generate an object of class metime\_analyser.
contains slots - list\_of\_data: For the list of all data matrices.
- list\_of\_col\_data: list of all the col data files in the same order.
- list\_of\_row\_data: list of all the row data files in the same order.
- annotations: list with phenotype and medication. Each of which is character that represents
the name of the aforementioned dataset types.
\end{Description}
\inputencoding{utf8}
\HeaderA{mod\_code\_metab\_names}{Function to convert metabolite names to IDs}{mod.Rul.code.Rul.metab.Rul.names}
%
\begin{Description}\relax
Function to convert metabolite names to IDs
\end{Description}
%
\begin{Usage}
\begin{verbatim}
mod_code_metab_names(object, which_data)
\end{verbatim}
\end{Usage}
%
\begin{Arguments}
\begin{ldescription}
\item[\code{object}] An S4 object of class metime\_analyser

\item[\code{which\_data}] character vector to define the datasets to use
\end{ldescription}
\end{Arguments}
%
\begin{Value}
A list with S4 object and list of mapping tables, the object can be used for GGMs
\end{Value}
\inputencoding{utf8}
\HeaderA{mod\_convert\_s4\_to\_s3}{Function to Convert S4 object of class metime\_analyser to an S3 object with same architecture}{mod.Rul.convert.Rul.s4.Rul.to.Rul.s3}
%
\begin{Description}\relax
converter function to be applied onto metime\_analyse object to convert into a standard list of S3 type.
\end{Description}
%
\begin{Usage}
\begin{verbatim}
mod_convert_s4_to_s3(object)
\end{verbatim}
\end{Usage}
%
\begin{Arguments}
\begin{ldescription}
\item[\code{object}] An object of class metime\_analyser
\end{ldescription}
\end{Arguments}
%
\begin{Value}
An S3 object of the same data as metime\_analyser in other words all slots are now converted into nested lists
\end{Value}
%
\begin{Examples}
\begin{ExampleCode}
# convert S4 object to a list
s3_list <- mod_convert_s4_to_s3(object=metime_analyser_object)
\end{ExampleCode}
\end{Examples}
\inputencoding{utf8}
\HeaderA{mod\_extract\_common\_samples}{Function to get only common samples from the dataframes in list\_of\_data}{mod.Rul.extract.Rul.common.Rul.samples}
%
\begin{Description}\relax
A method applied on object of class metime\_analyse to extract common samples across datasets. Also has an option to split the data according
timepoints(mod\_split\_acc\_time()).
\end{Description}
%
\begin{Usage}
\begin{verbatim}
mod_extract_common_samples(object, time_splitter = FALSE)
\end{verbatim}
\end{Usage}
%
\begin{Arguments}
\begin{ldescription}
\item[\code{object}] An object of class metime\_anaylser

\item[\code{time\_splitter}] A boolean input: True leads to splitting of the data wrt time,
False returns all the dataframes as they are with common rows
\end{ldescription}
\end{Arguments}
%
\begin{Value}
list\_of\_data with common samples across all time points
\end{Value}
%
\begin{Examples}
\begin{ExampleCode}
# extracting common samples across all datasets
new_list_of_data <- mod_common_sample_extractor(object=metime_analyser_object)
\end{ExampleCode}
\end{Examples}
\inputencoding{utf8}
\HeaderA{mod\_filter\_timepoints}{Functions for selecting time points}{mod.Rul.filter.Rul.timepoints}
%
\begin{Description}\relax
a method applied onto class metime\_analyser in order to extract timepoints of interest from a dataset
\end{Description}
%
\begin{Usage}
\begin{verbatim}
mod_filter_timepoints(object, timepoints, complete, which_data)
\end{verbatim}
\end{Usage}
%
\begin{Arguments}
\begin{ldescription}
\item[\code{object}] An object of class metime\_analyser

\item[\code{timepoints}] time points to be selected.

\item[\code{complete}] if TRUE subjects are only selected if measured in all selected time points

\item[\code{which\_data}] Name of the dataset to be used
\end{ldescription}
\end{Arguments}
%
\begin{Value}
An object of class metime\_analyser with processed data
\end{Value}
%
\begin{Examples}
\begin{ExampleCode}
#example to use this function
object <- mod_filter_tp(object, timepoints=c("t0","t12","t24"), full=TRUE, which_data="Name of the dataset")
\end{ExampleCode}
\end{Examples}
\inputencoding{utf8}
\HeaderA{mod\_merge\_data}{Function to merge two sets of data for any analysis}{mod.Rul.merge.Rul.data}
%
\begin{Description}\relax
A function to merge two or more datasets and use it for analysis
\end{Description}
%
\begin{Usage}
\begin{verbatim}
mod_merge_data(object, which_data, name)
\end{verbatim}
\end{Usage}
%
\begin{Arguments}
\begin{ldescription}
\item[\code{object}] An S4 object of class metime\_analyser

\item[\code{which\_data}] Datasets to be merged. Only two or more are allowed

\item[\code{name}] character vector to define the name of the new dataset
\end{ldescription}
\end{Arguments}
%
\begin{Value}
A new S4 object of class metime\_analyser with the new merged dataset appended to it
\end{Value}
\inputencoding{utf8}
\HeaderA{mod\_merge\_metime\_analysers}{Function to merge one or more metime\_analyser objects}{mod.Rul.merge.Rul.metime.Rul.analysers}
%
\begin{Description}\relax
function to merge multiple metime\_analyser objects
\end{Description}
%
\begin{Usage}
\begin{verbatim}
mod_merge_metime_analysers(list_of_objects, annotations_index)
\end{verbatim}
\end{Usage}
%
\begin{Arguments}
\begin{ldescription}
\item[\code{list\_of\_objects}] list of metime analyser objects that are to be merged

\item[\code{annotations\_index}] new list with annotations\_index. Can also set to be NULL.
\end{ldescription}
\end{Arguments}
%
\begin{Value}
A merged metime\_analyser object
\end{Value}
\inputencoding{utf8}
\HeaderA{mod\_merge\_results}{Function to combine plotter objects}{mod.Rul.merge.Rul.results}
%
\begin{Description}\relax
Function to combine plotter objects based on similar calc type
\end{Description}
%
\begin{Usage}
\begin{verbatim}
mod_merge_results(object, results_indices, groups)
\end{verbatim}
\end{Usage}
%
\begin{Arguments}
\begin{ldescription}
\item[\code{object}] An S4 object of class metime\_analyser

\item[\code{results\_indices}] one or more indices of results to merge data

\item[\code{groups}] character vector to define the groups that are involved
\end{ldescription}
\end{Arguments}
%
\begin{Value}
object with add merged letters
\end{Value}
\inputencoding{utf8}
\HeaderA{mod\_remove\_duplicates}{Function to remove duplicates}{mod.Rul.remove.Rul.duplicates}
%
\begin{Description}\relax
Function to remove duplicates from the analyser object
\end{Description}
%
\begin{Usage}
\begin{verbatim}
mod_remove_duplicates(object)
\end{verbatim}
\end{Usage}
%
\begin{Arguments}
\begin{ldescription}
\item[\code{object}] An S4 object of class metime\_analyser
\end{ldescription}
\end{Arguments}
%
\begin{Value}
object after removing duplicated data
\end{Value}
\inputencoding{utf8}
\HeaderA{mod\_remove\_nas}{Function to remove NA's from data matrices}{mod.Rul.remove.Rul.nas}
%
\begin{Description}\relax
A method applied on S4 object to remove NA's and change data accordingly
\end{Description}
%
\begin{Usage}
\begin{verbatim}
mod_remove_nas(object, which_data)
\end{verbatim}
\end{Usage}
%
\begin{Arguments}
\begin{ldescription}
\item[\code{object}] S4 object of class metime\_analyser

\item[\code{which\_data}] dataset/s for which the method is to be applied
\end{ldescription}
\end{Arguments}
%
\begin{Value}
S4 object with NA's removed and data manipulated accordingly
\end{Value}
\inputencoding{utf8}
\HeaderA{mod\_split\_acc\_to\_time}{Function to split data acoording to time}{mod.Rul.split.Rul.acc.Rul.to.Rul.time}
%
\begin{Description}\relax
Function to split the list of dataframes into a nested list with each dataframe
being split into into dataframes of different timepoints
\end{Description}
%
\begin{Usage}
\begin{verbatim}
mod_split_acc_to_time(object)
\end{verbatim}
\end{Usage}
%
\begin{Arguments}
\begin{ldescription}
\item[\code{object}] An object of class metime\_analyser
\end{ldescription}
\end{Arguments}
%
\begin{Value}
list\_of\_data with each dataframe being broken into a list of dataframes with respect to the timepoint they belong to
\end{Value}
%
\begin{Examples}
\begin{ExampleCode}
#splitting data according to time
new_data <- mod_split_acc_to_time(object=metime_analyser_object)
\end{ExampleCode}
\end{Examples}
\inputencoding{utf8}
\HeaderA{mod\_stratify\_analyser}{Function to stratify data in the metime analyser object}{mod.Rul.stratify.Rul.analyser}
%
\begin{Description}\relax
Function to stratify the data of interest into different objects that can be used
to perform calculations according the said stratification variable
\end{Description}
%
\begin{Usage}
\begin{verbatim}
mod_stratify_analyser(object, which_data, variable)
\end{verbatim}
\end{Usage}
%
\begin{Arguments}
\begin{ldescription}
\item[\code{object}] S4 object of class metime\_analyser

\item[\code{which\_data}] Dataset/datasets to be used for stratification

\item[\code{variable}] Phenotype based on which the stratification would be performed
\end{ldescription}
\end{Arguments}
%
\begin{Value}
list of metime\_analyser objects which are stratified based on the variable chosen
\end{Value}
\inputencoding{utf8}
\HeaderA{mod\_trans\_eigendata}{Function to add the clusters obtained from wgcna}{mod.Rul.trans.Rul.eigendata}
%
\begin{Description}\relax
A function to add cluster assiginment to the col\_data from WGCNA
\end{Description}
%
\begin{Usage}
\begin{verbatim}
mod_trans_eigendata(object, which_data, append, clusters, ...)
\end{verbatim}
\end{Usage}
%
\begin{Arguments}
\begin{ldescription}
\item[\code{object}] An S4 object of class metime\_analyser

\item[\code{which\_data}] character to define which dataset is to be used

\item[\code{append}] logical if set to true adds the new data to the object used else creates new object

\item[\code{clusters}] logical if set to true will add already existing cluster info otherwise creates new

\item[\code{...}] arguments for add\_clusters\_wgcna
\end{ldescription}
\end{Arguments}
%
\begin{Value}
metime\_analyser object with new dataset with eigendata of the metabolites
\end{Value}
\inputencoding{utf8}
\HeaderA{mod\_trans\_log}{Function to apply log transformation}{mod.Rul.trans.Rul.log}
%
\begin{Description}\relax
Function to log transform data
\end{Description}
%
\begin{Usage}
\begin{verbatim}
mod_trans_log(object, which_data, base)
\end{verbatim}
\end{Usage}
%
\begin{Arguments}
\begin{ldescription}
\item[\code{object}] An object of class metime\_analyser

\item[\code{which\_data}] Name of the dataset to be used

\item[\code{base}] base of log to be used
\end{ldescription}
\end{Arguments}
%
\begin{Value}
An object of class metime\_analyser with processed data
\end{Value}
%
\begin{Examples}
\begin{ExampleCode}
# example to apply log transformation
object <- mod_logtrans(object, which_data="name of the dataset", base=2)
\end{ExampleCode}
\end{Examples}
\inputencoding{utf8}
\HeaderA{mod\_trans\_zscore}{Function to scale the data}{mod.Rul.trans.Rul.zscore}
%
\begin{Description}\relax
Functions for scaling
\end{Description}
%
\begin{Usage}
\begin{verbatim}
mod_trans_zscore(object, which_data)
\end{verbatim}
\end{Usage}
%
\begin{Arguments}
\begin{ldescription}
\item[\code{object}] An object of class metime\_analyser

\item[\code{which\_data}] Name of the dataset to be used
\end{ldescription}
\end{Arguments}
%
\begin{Value}
An object of class metime\_analyser with processed data
\end{Value}
%
\begin{Examples}
\begin{ExampleCode}
# example to apply scaling
object <- mod_zscore(object, which_data="name of the dataset")
\end{ExampleCode}
\end{Examples}
\inputencoding{utf8}
\HeaderA{plot}{Setting a plotting method for the metime\_analyser class}{plot}
%
\begin{Description}\relax
Function to plot results of a certain calculation
\end{Description}
%
\begin{Usage}
\begin{verbatim}
plot(object, results_index, interactive, ...)
\end{verbatim}
\end{Usage}
%
\begin{Arguments}
\begin{ldescription}
\item[\code{object}] An S4 object of class metime\_analyser

\item[\code{results\_index}] Index/name of the results to be plotted

\item[\code{interactive}] logical. Set TRUE for interactive plot

\item[\code{...}] other parameters to pass color, fill, strat, viz(character vector with colnames for interactive)
\end{ldescription}
\end{Arguments}
%
\begin{Value}
plots for a certain set of results
\end{Value}
\inputencoding{utf8}
\HeaderA{save\_analyser\_object}{Function to extract analyser object data into a csv}{save.Rul.analyser.Rul.object}
%
\begin{Description}\relax
extracts information from analyser object and saves it as a csv
\end{Description}
%
\begin{Usage}
\begin{verbatim}
save_analyser_object(object, which_data, type)
\end{verbatim}
\end{Usage}
%
\begin{Arguments}
\begin{ldescription}
\item[\code{object}] An object of class metime\_plotter

\item[\code{which\_data}] Character to specify the dataset

\item[\code{type}] which type of output file. Can be "csv", "tsv" and "xlsx"
\end{ldescription}
\end{Arguments}
%
\begin{Value}
saves the data in the working directory as a csv and returns nothing
\end{Value}
%
\begin{Examples}
\begin{ExampleCode}
see examples here
save_analyser_object(object, which_data="dataset")
\end{ExampleCode}
\end{Examples}
\inputencoding{utf8}
\HeaderA{save\_results}{Function to extract results into different types of files}{save.Rul.results}
%
\begin{Description}\relax
extracts results from analyser object and saves it
\end{Description}
%
\begin{Usage}
\begin{verbatim}
save_results(object, results_index, type)
\end{verbatim}
\end{Usage}
%
\begin{Arguments}
\begin{ldescription}
\item[\code{object}] An object of class metime\_analyser

\item[\code{results\_index}] character or numeric to define the results of interest

\item[\code{type}] character to define outfile type that is "csv", "xlsx" or "tsv"
\end{ldescription}
\end{Arguments}
%
\begin{Value}
saves the data into a csv and returns nothing
\end{Value}
%
\begin{Examples}
\begin{ExampleCode}
see examples here and maintain the order
If type is xlsx one out file is enough and the network data will be stored in different sheets
Network : save_results(object, results_index, out=c("edge", "node", "meta"), type="csv")
Others : save_results(object, results_index, out=c("outfile1", ...), type="tsv")
\end{ExampleCode}
\end{Examples}
\inputencoding{utf8}
\HeaderA{set\_parallel\_cores}{register parallel backend}{set.Rul.parallel.Rul.cores}
%
\begin{Description}\relax
function to run in order to perform the analysis parallely thereby saving time
\end{Description}
%
\begin{Usage}
\begin{verbatim}
set_parallel_cores(n_cores = NULL)
\end{verbatim}
\end{Usage}
%
\begin{Arguments}
\begin{ldescription}
\item[\code{n\_cores}] A number of specified cores.
\end{ldescription}
\end{Arguments}
%
\begin{Value}
set a parallel backend
\end{Value}
\inputencoding{utf8}
\HeaderA{show,metime\_analyser-method}{Setting new print definition for the metime\_analyser object}{show,metime.Rul.analyser.Rdash.method}
%
\begin{Description}\relax
function to see the structure of metime\_analyser object
\end{Description}
%
\begin{Usage}
\begin{verbatim}
## S4 method for signature 'metime_analyser'
show(object)
\end{verbatim}
\end{Usage}
%
\begin{Arguments}
\begin{ldescription}
\item[\code{object}] S4 object of class metime\_analyser
\end{ldescription}
\end{Arguments}
%
\begin{Value}
structure of the S4 object
\end{Value}
%
\begin{Examples}
\begin{ExampleCode}
structure(object)
\end{ExampleCode}
\end{Examples}
\inputencoding{utf8}
\HeaderA{structure}{Setting new structure definition for the metime\_analyser object}{structure}
%
\begin{Description}\relax
function to see the structure of metime\_analyser object
\end{Description}
%
\begin{Usage}
\begin{verbatim}
structure(object)
\end{verbatim}
\end{Usage}
%
\begin{Arguments}
\begin{ldescription}
\item[\code{object}] S4 object of class metime\_analyser
\end{ldescription}
\end{Arguments}
%
\begin{Value}
structure of the S4 object
\end{Value}
%
\begin{Examples}
\begin{ExampleCode}
structure(object)
\end{ExampleCode}
\end{Examples}
\inputencoding{utf8}
\HeaderA{validity}{Validity function to check if the object is valid or not}{validity}
%
\begin{Description}\relax
Function to check the validity of metime\_analyser
\end{Description}
%
\begin{Usage}
\begin{verbatim}
validity(object)
\end{verbatim}
\end{Usage}
%
\begin{Arguments}
\begin{ldescription}
\item[\code{object}] An S4 object of class metime\_analyser
\end{ldescription}
\end{Arguments}
%
\begin{Value}
logical suggesting if the object is intact or not
\end{Value}
%
\begin{Examples}
\begin{ExampleCode}
validObject(object)
\end{ExampleCode}
\end{Examples}
\inputencoding{utf8}
\HeaderA{viz\_distribution\_plotter}{Function for Plotting distributions of phenotypic variables}{viz.Rul.distribution.Rul.plotter}
%
\begin{Description}\relax
A method to be applied onto s4 object so as to obtain distributions of various phenotypic variables
\end{Description}
%
\begin{Usage}
\begin{verbatim}
viz_distribution_plotter(object, colname, which_data, strats, phenotype)
\end{verbatim}
\end{Usage}
%
\begin{Arguments}
\begin{ldescription}
\item[\code{object}] An object of class metime\_analyser

\item[\code{colname}] Name of the variable whose distribution is of interest

\item[\code{which\_data}] Name of the dataset from which the samples will be extracted

\item[\code{strats}] Character vector with colnames that are to be used for stratification

\item[\code{phenotype}] Logical. If true data will be collected from phenotype\_data matrix else from row data
\end{ldescription}
\end{Arguments}
%
\begin{Value}
a list with either 1) density plot, mean table acc to timepoint and variable type or
2) bar plot, line plot, and variable type
\end{Value}
%
\begin{Examples}
\begin{ExampleCode}
# extracting distribuiton of Age from dataset1
plot <- viz_distribution_plotter(object, colname="Age", which_data="dataset1", strats="additional columns for facet wrapping")
\end{ExampleCode}
\end{Examples}
